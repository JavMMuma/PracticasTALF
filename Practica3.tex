\documentclass{article}
\usepackage{lmodern}
\usepackage[T1]{fontenc}
\usepackage[spanish,activeacute]{babel}
\usepackage{mathtools}
\usepackage{tikz}
\usepackage{array}
\usepackage{ragged2e}
\usepackage{amsmath}
\usepackage{graphicx}
\graphicspath{{./}}

\title{\Large \bf Teoría de Autómatas y Lenguajes Formales

\LARGE \rm Práctica III}
\author{Javier Molina Montiel}

\begin{document}
\maketitle
\justify\normalsize
Actividad 1:

La turing machine solución de este ejercicio es:
\[
\begin{bmatrix}
0 & * & l & 1\\
0 & | & | & 0\\
1 & * & | & 2\\
1 & | & l & 1\\
2 & * & l & 3\\
2 & | & r & 2\\
3 & * & l & 4\\
3 & | & * & 3\\
4 & * & h & 4\\
4 & | & * & 4
\end{bmatrix}
\]
\includegraphics[width = 13cm, height = 7cm]{tm1}

\justify\normalsize
Actividad 2:

La funcion recursiva para sumar tres números es:

$<<\pi_1^1 | \sigma (\pi_3^3) > | \sigma (\pi_4^4) >$
%<<π^1_1|σ(π^3_3)>|σ(π^4_4)>
\justify\normalsize
\includegraphics[width = 11cm, height = 7.5cm]{capturaoctave}
\justify\normalsize
Actividad 3:

additionthree = (3,s)

s:

\quad while $X_2$ != 0 do 

\quad\quad $X_1$ := $X_1$  + 1;

\quad\quad $X_2$ := $X_2$  - 1

\quad od

\quad \quad while $X_3$ != 0 do 

\quad\quad\quad $X_1$ := $X_1$  + 1;

\quad\quad\quad $X_2$ := $X_3$  - 1

\quad\quad od

\quad\quad\quad $X_1$ := $X_1$

\end{document}